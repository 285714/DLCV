\documentclass[11pt]{article} %[12pt]

\usepackage[utf8]{inputenc}
\usepackage[T1]{fontenc}
\usepackage{lmodern}
\usepackage[ngerman]{babel}
\usepackage{amsmath}
\usepackage{listings}
\usepackage{graphicx}
\usepackage{enumitem}
\usepackage{amssymb}
\usepackage{verbatim}
\usepackage{pdfpages}
\usepackage{algpseudocode}
\usepackage{breqn}
\usepackage[linesnumbered,ruled,vlined]{algorithm2e}
\usepackage{tikz}
\usetikzlibrary{positioning,shapes.multipart}
\usepackage{textcomp}
\SetArgSty{textup}
\usepackage{fancyhdr}

\topmargin -2cm 
\textheight 24cm
\textwidth 16.0 cm 
\oddsidemargin -0.1cm


 \pagestyle{fancy}
 \fancyhf{}
 \fancyhead[LE,RO]{Blatt 1 \quad \today{}}
 \fancyhead[RE,LO]{Fabian Späh \quad B5}
 \fancyfoot[CE,CO]{\leftmark}
 \fancyfoot[LE,RO]{\thepage}

\newcommand{\R}{\mathbb{R}}
\newcommand{\N}{\mathbb{N}}
\newcommand{\Q}{\mathbb{Q}}
\newcommand{\C}{\mathbb{C}}
\newcommand{\Z}{\mathbb{Z}}
\newcommand{\V}{\mathcal{V}}
\newcommand{\ggT}{\text{ggT}}
\newcommand{\kgV}{\text{kgV}}
\newcommand{\im}{\text{im} \, }
\newcommand{\MinPol}{\text{MinPol}}
\newcommand{\spann}{\text{span}}
\newcommand{\spa}[1]{\; #1 \;}
\newcommand{\Sp}{\mathrm{Sp}}
\newcommand{\set}[1]{\{\, {#1} \,\}}
\newcommand{\bewbeh}[2]{\begin{enumerate} \item[\textbf{Beh.}]#1 \item[\textbf{Bew.}]#2
	\end{enumerate}}
\newcommand{\induct}[2]{\begin{enumerate} \item[\textit{IA}]#1 \item[\textit{IS}]#2
	\end{enumerate}}
\newcommand{\qed}{\hfill\ensuremath{\square}}
\newcommand{\gdw}[2]{\begin{enumerate} \item[$\Rightarrow$]#1 \item[$\Leftarrow$]#2
	\end{enumerate}}
\newcommand{\intsto}[1]{\{1, \dots, #1\}}
\newcommand\Item[1][]{%
	\ifx\relax#1\relax  \item \else \item[#1] \fi
	\abovedisplayskip=0pt\abovedisplayshortskip=0pt~\vspace*{-\baselineskip}}



\begin{document}

If the gradient of $f$ exists, it holds for every $\tau, q$ that
\[
  f(\mathbf p + \tau q) = f(\mathbf p) + \tau \nabla f(\mathbf p)^T \cdot q
  + r(\tau)
\]
with $\lim_{\tau \to 0} \frac{r(\tau)}{\tau} = 0$.
Hence we get for $h$
\[
  \begin{array}{rl}
    & h(\mathbf p + \tau q) = g(f(\mathbf p+ \tau q) =
    g(f(\mathbf p) + \tau \nabla f(\mathbf p)^T \cdot q + r(\tau)) \\
    =& g(f(\mathbf p) + \tau (\nabla f(\mathbf p)^T \cdot q + \frac{r(\tau)}{\tau})) \\
    =& g(f(\mathbf p)) + \tau g'(f(\mathbf p)) \cdot (\nabla f(\mathbf p)^T
      \cdot q + \frac{r(\tau)}{\tau}) + r'(\tau) \\
    =& h(\mathbf p) + \tau g'(f(\mathbf p)) \cdot \nabla f(\mathbf p)^T
      \cdot q + \tau g'(f(\mathbf p)) \frac{r(\tau)}{\tau} + r'(\tau) \ .
  \end{array}
\]
It clearly is
\[
  \lim_{\tau \to 0} \frac{\tau g'(f(\mathbf p)) \frac{r(\tau)}{\tau} + r'(\tau)}{\tau}
  = \lim_{\tau \to 0} g'(f(\mathbf p)) \cdot \frac{r(\tau)}{\tau} + \frac{r'(\tau)}{\tau}
  = 0
\]
and thus
\[
  h(\mathbf p + \tau q) = h(\mathbf p) + \tau \nabla h(\mathbf p)^T \cdot q
  + r''(\tau)
\]
with $r''(\tau) = \tau g'(f(\mathbf p)) \frac{r(\tau)}{\tau} + r'(\tau)$
and $\nabla h(\mathbf p) = g'(f(\mathbf p)) \cdot \nabla f(\mathbf p)$
which was to show.




\end{document}
