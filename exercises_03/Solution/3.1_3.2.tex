\documentclass[10pt]{article} %[12pt]

\usepackage[utf8]{inputenc}
\usepackage[T1]{fontenc}
\usepackage{lmodern}
\usepackage[ngerman]{babel}
\usepackage{amsmath}
\usepackage{listings}
\usepackage{graphicx}
\usepackage{enumitem}
\usepackage{amssymb}
\usepackage{verbatim}
\usepackage{pdfpages}
\usepackage{algpseudocode}
\usepackage{breqn}
\usepackage[linesnumbered,ruled,vlined]{algorithm2e}
\usepackage{tikz}
\usetikzlibrary{positioning,shapes.multipart}
\usepackage{textcomp}
\SetArgSty{textup}
\usepackage{fancyhdr}

\topmargin -2cm 
\textheight 24cm
\textwidth 16.0 cm 
\oddsidemargin -0.1cm


 \pagestyle{fancy}
 \fancyhf{}
 \fancyhead[LE,RO]{Blatt 3 \quad \today{}}
 \fancyhead[RE,LO]{Fabian Späh \quad B5}
 \fancyfoot[CE,CO]{\leftmark}
 \fancyfoot[LE,RO]{\thepage}

\newcommand{\R}{\mathbb{R}}
\newcommand{\N}{\mathbb{N}}
\newcommand{\Q}{\mathbb{Q}}
\newcommand{\C}{\mathbb{C}}
\newcommand{\Z}{\mathbb{Z}}
\newcommand{\V}{\mathcal{V}}
\newcommand{\ggT}{\text{ggT}}
\newcommand{\kgV}{\text{kgV}}
\newcommand{\im}{\text{im} \, }
\newcommand{\MinPol}{\text{MinPol}}
\newcommand{\spann}{\text{span}}
\newcommand{\spa}[1]{\; #1 \;}
\newcommand{\Sp}{\mathrm{Sp}}
\newcommand{\set}[1]{\{\, {#1} \,\}}
\newcommand{\bewbeh}[2]{\begin{enumerate} \item[\textbf{Beh.}]#1 \item[\textbf{Bew.}]#2
	\end{enumerate}}
\newcommand{\induct}[2]{\begin{enumerate} \item[\textit{IA}]#1 \item[\textit{IS}]#2
	\end{enumerate}}
\newcommand{\qed}{\hfill\ensuremath{\square}}
\newcommand{\gdw}[2]{\begin{enumerate} \item[$\Rightarrow$]#1 \item[$\Leftarrow$]#2
	\end{enumerate}}
\newcommand{\intsto}[1]{\{1, \dots, #1\}}
\newcommand\Item[1][]{%
	\ifx\relax#1\relax  \item \else \item[#1] \fi
	\abovedisplayskip=0pt\abovedisplayshortskip=0pt~\vspace*{-\baselineskip}}



\begin{document}


\section*{Exercise 1}

We can identify $g$ as a function $g : \R^{M+M} \to \R$ and write $h : \R^N \to \R$
equivalently as $h = g \circ \tilde f$ where $\tilde f : \R^N \to \R^{M+M}$ is
defined as $\tilde f(x) = [ f(x), f(x) ]^T$.
Every column $n \in \{1, \dots, N\}$ in jacobian of $\tilde f$ equals
\[
  \begin{array}{r}
    D_i \tilde f (x) = \frac{\partial \tilde f}{\partial x_1}
    = \left[
      \frac{\partial \tilde f_1}{\partial x_i}(x),
      \dots,
      \frac{\partial \tilde f_M}{\partial x_i}(x),
      \frac{\partial \tilde f_{M+1}}{\partial x_i}(x),
      \dots,
      \frac{\partial \tilde f_{M+M}}{\partial x_i}(x)
    \right]^T \qquad \qquad \qquad \qquad \qquad \qquad \\
    = \left[
      \frac{\partial f_1}{\partial x_i}(x),
      \dots,
      \frac{\partial f_M}{\partial x_i}(x),
      \frac{\partial f_1}{\partial x_i}(x),
      \dots,
      \frac{\partial f_M}{\partial x_i}(x)
    \right]^T
    = [D_i f(x), D_i f(x)]^T
  \end{array}
\]
and hence $D \tilde f(x) = [D f(x), D f(x)]^T \in \R^{(M+M) \times N}$.
This is sufficient for
\[
  Dh(x) = Dg(\tilde f(x)) \cdot D \tilde f(x)
  = Dg([f(x), f(x)]^T) \cdot [Df(x), Df(x)]^T
\]


\section*{Exercise 2}

It is
\[
  \frac{\partial E}{\partial w}(\mathbf w)
  = \frac 1 L \sum_{l=1}^L 2 \cdot (d_l - f(\mathbf x_l; \mathbf w)) \cdot
    \frac{\partial f}{\partial w}(\mathbf w)
\]
and
\[
  \frac{\partial f}{\partial w^{2,0}_{1,1}}(\mathbf x; \mathbf w)
  = \sum_{j=0}^2 w_{0,j}^{1,1} \cdot x_j, \qquad
  \frac{\partial f}{\partial w^{1,0}_{0,1}}(\mathbf x; \mathbf w)
  = w^{2,0}_{1,0} \cdot x_1
\]
for $\mathbf x = [x_0, x_1, x_2]^T$.




\end{document}
